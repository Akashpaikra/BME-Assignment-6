\documentclass[12pt]{article}
\usepackage[english]{babel}
\usepackage{natbib}
\usepackage{url}
\usepackage[utf8x]{inputenc}
\usepackage{amsmath}
\usepackage{graphicx}
\graphicspath{{NITLOGO/}}
\usepackage{parskip}
\usepackage{fancyhdr}
\usepackage{vmargin}
\setmarginsrb{3 cm}{2.5 cm}{3 cm}{2.5 cm}{1 cm}{1.5 cm}{1 cm}{1.5 cm}

\title{5 solutions to covid19 provided by biomedical engineers }\newline \\\\				
\author{21111006}								
\date{28 JAN 2022}						
\makeatletter
\let\thetitle\@title
\let\theauthor\@author
\let\thedate\@date
\makeatother

\pagestyle{fancy}
\fancyhf{}
\rhead{\theauthor}
\lhead{\thetitle}
\cfoot{\thepage}

\begin{document}
\begin{titlepage}
	\centering
    \includegraphics[scale = 0.22]{NITLOGO}\\[1.0 cm]	
    \textsc{\LARGE National Institute Of Technology \newline\\\\ RAIPUR}\\[2.0 CM]
    
	\textsc{\Large ASSIGNMENT 06}\\[0.5 cm]				% Course Code
	\rule{\linewidth}{0.4 mm} \\[0.4 cm]
	{ \huge \bfseries \thetitle}\\
	\rule{\linewidth}{0.4 mm} \\[1.5 cm]
	
	\begin{minipage}{0.6\textwidth}
		\begin{flushleft} \large
			\emph{Submitted To:}\\
			Mr. Saurabh Gupta\\
            Department Of Basic Biomedical Engineering\\
			\end{flushleft}
			\end{minipage}~
			\begin{minipage}{0.4\textwidth}
            
			\begin{flushright} \large
			\emph{Submitted By :}\\
			Akash Paikra\\
            21111006\\
		\end{flushright}
        
	\end{minipage}\\[2 cm]
\end{titlepage}

\tableofcontents
\pagebreak







\section{INTRODUCTION:}


\subsection{WHAT TO DO BIOMEDICAL ENGINEERS ?}

A Biomedical Engineer's job include designing biomedical equipment and devices to help people recover or enhance their health. Internal devices, such as stents or artificial organs, as well as external devices, such as braces and supports, are examples (orthotics). It may also entail the design and adaptation of medical equipment. It's a job that necessitates a strong understanding of computing, biology, and engineering, as well as a creative mindset and problem-solving abilities.



Biomedical engineering is a hot topic right now, especially with the COVID-19 virus sweeping around the world. Biomedical engineering (sometimes referred to as "medical engineering") is a term that refers to the application of engineering materials to medicine and healthcare. Biomedical engineering is critical for the healthcare sector – from developing new medical treatments to monitoring a patient's condition – and without it, the industry would be extremely unreliable.




\subsection{JUSTIFICATION :}


In the COVID pandemic, the usage of medical equipment is an awful indicator that the patients are suffering from acute respiratory distress and require aid to breathe.


\section{TREATMENT MECHANISMS :}


\subsection{OXYGEN THERAPY :}
The delivery of additional oxygen using a nasal cannula or a more intrusive face mask is usually the primary form of treatment for mild respiratory insufficiency. The oxygen is usually delivered in cylinders, which are either tiny for transportation or big for fixed patients and longer-term supplies.

Although oxygen concentrators are an appealing option to tanks, they are rarely used in hospital settings for caring for COVID-19 patients. Oxygen concentrators take oxygen from the air and deliver it to the patient on demand. Concentrators are available in a variety of sizes, ranging from a small portable shoulder bag to larger fixed units for patients who require oxygen 24 hours a day.


High flow nasal oxygen (HFNO) is a type of oxygen delivery that delivers warmed and humidified oxygen at high flow rates (usually tens of litres/min) at body temperature and up to 100 percent RH and 100 percent oxygen to avoid drying out the airways.




\subsection{CONTINUOUS POSITIVE AIRWAY PRESSURE (CPAP):}

Continuous Positive Airway Pressure (CPAP) is the next step in treating COVID-19 patients. CPAP was originally designed to avoid airways collapse in sleep apnoea patients, but it has been demonstrated to be beneficial to COVID patients if used early enough in the disease's course.


A well-fitting face mask is an important part of a CPAP machine, but it may be rather bothersome. Because CPAP effectively resists some resistance to expiration, it is only suited for patients who are capable of some breathing strength. There are variants that adapt the level of pressure automatically to the patient's breathing characteristics (APAP) or have distinct levels of pressure for inspiration and expiration (BiPAP). CPAP normally provides the patient with (filtered) air, but most masks feature a port for adding oxygen to the mix.



\subsection{VENTILATORS :}


Patients who are unable to breathe on their own must be placed on a ventilator. Patients in an advanced stage of respiratory distress are frequently intubated and sedated at the start of treatment since ventilators can replace breath function.



Patients in an advanced stage of respiratory distress are frequently intubated and sedated at the start of treatment since ventilators can replace breath function. They are complicated devices that give healthcare providers a lot of flexibility in terms of adjusting assisted breathing settings and eventually weaning healing patients off the ventilator.




Modern ventilators are usually pressure-controlled closed loops that can detect spontaneous breathing and provide synchronised aid to recovering patients. They also allow the patient to alter the composition of the gas he or she breathes, ranging from regular air to 100 percent oxygen. They normally get their supply from the hospital's gas supply network, but they can also be connected to oxygen tanks or oxygen concentrators if there isn't one.



\subsection{PATIENT MONITORING :}

One of the key parameters for COVID-19 patient is the amount of oxygen in their bloodstream (SpO2), measured by pulse oximetry which uses optics within a finger clamp. Pulse oximetry tends to be used for the duration of the patient’s stay in ICU.

Modern patient monitors provide many more patient parameters all the way to breathing waveforms to enable clinicians to fine tune their care of the patients.


The amount of oxygen in a patient's bloodstream (SpO2), which is evaluated by pulse oximetry, which uses optics within a finger clamp, is one of the most important metrics for COVID-19 patients. Pulse oximetry is often used for the duration of a patient's stay in the intensive care unit.

Modern patient monitors provide a plethora of additional patient parameters, all the way down to breathing waveforms, allowing doctors to fine-tune their patient treatment.


\section{MANAGEMENT OF THE PANDEMIC IN HOSPITALS:}


\subsection{PERSONAL PROTECTIVE EQUIPMENT :}
 
 
 The COVID-19 epidemic has highlighted society's vulnerability and the necessity for comprehensive and practical protection measures. Face masks, as personal protection equipment (PPE), remain the greatest practicable line of defence against SARS-CoV-2 and other respiratory virus illnesses for the general public.
 
 
 
 However, greater protection is required for a wide range of interdisciplinary health care personnel, such as surgical or respirator masks, which are not designed to be worn for as long as an NHS shift requires. These disposable objects have an environmental cost, do not fit all face shapes, the mask-face seal can be broken while talking, and put pressure on the sensitive face skin, causing discomfort and tissue injury.
 
 As a result, there is a pressing need to develop new practical PPE technologies that can protect people while maintaining society's functionality. Several groups of engineers have been working on improved personal protective equipment (PPE), such as powered air purifying respirators (PAPRs), which are similar to the commercially available devices that were in short supply or unavailable during the onset of the epidemic.
 
 
 
 These have the following advantages:

being potentially more protective, as all breathed air is filtered because they include a hood that protects the face from droplets and self-infection by touching; avoiding mechanical loads on the delicate facial skin being cleanable and reusable, resulting in a lower potential cost and less plastic waste being a more inclusive device, as they should fit all users and permit lip reading and the care benefits of seeing your healthcare worker's face being a more inclusive device, as they should fit all users and permit lip reading and the care.


\subsection{ROLE OF CLINICAL ENGINEERS :}

Clinical engineers play a critical role in the use of technology in patient care, from procurement through maintenance, as well as, and this is less well known, collaborating with clinicians to develop innovative technologies that enable novel therapies.

\subsection{USER REQUIREMENTS :}

Due to a scarcity of medical equipment and personal protective equipment, governments rushed to purchase and stockpile working items, creating problems for their development and production. Certified medical device manufacturers, individuals, and teams of engineers put together hundreds of projects in quick speed. Information and designs were exchanged online in a truly collaborative effort to develop solutions that could be critical in combating the pandemic.



Despite the commendable effort and the several successful ideas that resulted from it, many gadgets that appeared to be technically sound on paper and cost-effective for mass manufacture may not have been up to par. Teams of engineers without clinical knowledge in highly infectious disease wards and intensive care units may have missed some user requirements and human variables. Coronavirus patients, for example, require specialised, repeatable, and controlled breathing that is compatible with the several systems used in critical care, such as kidney filtration devices and medicine delivery systems.



\section{PUBLIC-PRIVATE PARTNERSHIPS DURING THE COVID-19 PANDEMIC:}

During the COVID-19 pandemic, innovative public-private partnerships arose to compensate for and move the country away from the disease's high morbidity and mortality. These collaborations were ground-breaking and yielded major results, since they resulted in a vaccine being designed, tested, and licenced in just 11 months, an incredible feat. They also provided critical lessons learnt for future pandemic preparedness, as gaps in executing a coordinated, cross-sectoral approach of the magnitude, scope, and complexity required to meet the growing demands and trends remained.

Preclinical, in charge of expanding access to animal models and identifying useful tests;
Clinical Therapeutics is in charge of prioritising and testing new therapeutics, as well as developing master protocols for clinical studies.
Clinical trial capacity, which is in charge of creating survey instruments, compiling a list of clinical trial networks, and directing the implementation of innovative solutions; & Vaccines, which is in charge of expediting vaccine candidate evaluation, developing biomarkers to expedite approval, and presenting data to resolve safety concerns.


\section{PCR-KITS :}

Pandemic-prone infections, such as Covid, are being studied using polymerase chain response (PCR), a sub-atomic method. Primerdesign and Kogene Biotech, for example, have provided PCR kits that can aid in the diagnosis of the condition. In China, there has been a shortage of Covid test units, causing analyses to be delayed, and some are stating that acquiring one is like "winning the lottery."



















































\end{document}